\documentclass[a0,portrait]{a0poster}
\usepackage{multicol}
\usepackage{poster}
\usepackage{amsmath,amsthm,amssymb}
\usepackage{epstopdf}
\usepackage{xspace}
%\usepackage{sagetex}
\usepackage{circuitikz}

\newcommand{\Bold}{\mathbf}

\title{Homomorphic Cryptosystems}

\author{
Jeremy Caci, Ali Hajy, Jonah Jolley\\
Clark Rinker, Philip Robinson
}


\begin{document}
\maketitle

\def\fh{{\em Fully Homomorphic\xspace}}

\begin{multicols}{2}
\begin{slide}{Abstract}

We present a formal inquiry of the first fully homomorphic encryption scheme proposed by Craig Gentry in 2008. Moreover we introduce the first implementation of this cipher in the Python programming language. A fully homomorphic encryption scheme enables the execution of arbitrary operations on encrypted data without the decryption key which, put simply, allows for a third party to  store and manipulate sensitive information without the ability to interpret it. 
\parskip 1em

The applications of an efficient fully homomorphic encryption system are potentially limitless. A contemporary example is cloud based email, where a decentralized server stores and serves encrypted user email. Furthermore, such a system would allow for users to request the server to perform search queries on stored data without loss of privacy. This is a particularly enticing scenario given the recent boom in portable devices and multiparty computation, both of which currently have serious security concerns. 

Our investigation includes a high level synopsis of the mathematics involved in fully homomorphic encryption, a complete demonstration of our implementation in Python, and finally a overview of the space and asymptotic time complexities of the proposed system. 

\end{slide}

\begin{slide}{Definitions}
In order for a cryptographic system to be considered \fh we need to make some clear definitions availible. It also proves benificial to clearly express why these terms are significant and their relationships to eachother. 

\begin{itemize}
\item Circuit Operations
\begin{itemize}
\item Fundamental Gates
\item Polynomials
\end{itemize}
\item Bootstrapping
\begin{itemize}
\item Compact
\item Circular Security
\end{itemize}
\item Secure Problems
\begin{itemize}
\item Approximate GCD
\item Subset Sum Problem
\item Computational Complexity
\end{itemize}

\end{itemize}


We may need to additionally add information about Rings, Prime Ideals, this can be added later. There are likely other definitions that are needed, and will likely be found as the ones above are addressed
\end{slide}

\begin{slide}{Circuit probably moved}
\begin{circuitikz}\draw
(0,0) node[and port,scale=2.5] (myand) {}
(myand.in 1) node[anchor=east] {\({b_0}\)}
(myand.in 2) node[anchor=east] {\({b_1}\)}
(myand.out) node[anchor=west] {\({b_0 \cdot b_1}\;(\text{mod }2)\)}
;\end{circuitikz}

\begin{circuitikz}\draw
(0,0) node[xor port,scale=2.5] (myand) {}
(myand.in 1) node[anchor=east] {\({b_0}\)}
(myand.in 2) node[anchor=east] {\({b_1}\)}
(myand.out) node[anchor=west] {\({b_0 + b_1}\;(\text{mod } 2)\)}
;\end{circuitikz}

\begin{circuitikz}\draw
(0,0) node[not port,scale=2.5] (not) {}
(not.in) node[anchor=east] {\({b_0}\)}
(not.out) node[anchor=west] {\(1-{b_0}\;(\text{mod } 2)\)}
;\end{circuitikz}
\end{slide}

\begin{slide}{Security}

Given \(\lambda\) as a security Parameter 

we have a private key space of \(2^{(\lambda^2-2)} - (2^{\lambda^{2-1}})-1\)

The cryptographic hint uses the subset sum problem over a sparse subset. cracking the hint takes up space 

\[\left(\begin{matrix}\beta = \lambda^5\\\alpha \end{matrix}\right)\approx \beta^\alpha\] produces


The problem of cracking is {\em Soft NP-Complete}

This is because as the security parameters grows, the searchspace and the cost of the operations grow. 
\end{slide}

\end{multicols}
\begin{multicols}{3}
\begin{slide}{Mathematical Background}
Describing the abstract Ideals relations
\end{slide}

\begin{slide}{Bootstrapping}
The Power of Bootstraping
\end{slide}

\begin{slide}{System Resources}
Memory Used for Project \\ Time to complete \(O(n)\) equivelent operation \\ Projected Space \\ Keysize 
\end{slide}

\begin{slide}{Significant Difficulties}

\end{slide}

\begin{slide}{Conclusion}
Our Findings \\ Further Implementations \\ Python as a Vehicle
\end{slide}

\begin{slide}{Bibliography}
Thats where the bibliography goes
\end{slide}
\end{multicols}
\end{document}
