\documentclass[a0,portrait]{a0poster}
\usepackage{multicol}
\usepackage{poster}
\usepackage{amsmath,amsthm,amssymb}
\usepackage{epstopdf}
\usepackage{xspace}
%\usepackage{sagetex}

\newcommand{\Bold}{\mathbf}

\title{Homomorphic Cryptosystems}

\author{
Jeremy Caci, Ali Hajy, Jonah Jolly\\
Clark Rinker, Philip Robinson
}


\begin{document}
\maketitle



\begin{multicols}{2}
\begin{slide}{Abstract}

We present a formal inquiry of the first fully homomorphic encryption scheme proposed by Craig Gentry in 2008. Moreover we introduce the first implementation of this cipher in the Python programming language. A fully homomorphic encryption scheme enables the execution of arbitrary operations on encrypted data without the decryption key which, put simply, allows for a third party to  store and manipulate sensitive information without the ability to interpret it. 
\parskip 1em

The applications of an efficient fully homomorphic encryption system are potentially limitless. A contemporary example is cloud based email, where a decentralized server stores and serves encrypted user email. Furthermore, such a system would allow for users to request the server to perform search queries on stored data without loss of privacy. This is a particularly enticing scenario given the recent boom in portable devices and multiparty computation, both of which currently have serious security concerns. 

Our investigation includes a high level synopsis of the mathematics involved in fully homomorphic encryption, a complete demonstration of our implementation in Python, and finally a overview of the space and asymptotic time complexities of the proposed system. 

\end{slide}

\begin{slide}{Homomorphic Encryption}
 \centerline{description homomorphic encryption \\ With images and diagrams}

%\(\sage{3 + 4}\)
\end{slide}

\end{multicols}
\begin{multicols}{3}
\begin{slide}{Mathematical Background}
Describing the abstract Ideals relations
\end{slide}

\begin{slide}{Bootstrapping}
The Power of Bootstraping
\end{slide}

\begin{slide}{System Resources}
Memory Used for Project \\ Time to complete \(O(n)\) equivelent operation \\ Projected Space \\ Keysize 
\end{slide}

\begin{slide}{Significant Difficulties}

\end{slide}

\begin{slide}{Conclusion}
Our Findings \\ Further Implementations \\ Python as a Vehicle
\end{slide}

\begin{slide}{Bibliography}
Thats where the bibliography goes
\end{slide}
\end{multicols}
\end{document}
